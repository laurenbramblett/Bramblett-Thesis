\section{Conclusion}
The proposed estimation methodology uses aircraft performance characteristics. The average difference between the tested NASIC model and the developed methodology showed that the estimation technique is effective and the tradeoff between performance metrics can be quickly estimated with one run of the model for a combination of any two of the metrics, range, loiter time, or loiter altitude, with the third held constant. This efficiency in runtime and estimation of tradeoffs is far superior to the original NASIC method which involved running individual points and linearly interpolating between results. The new method can be used to account for flight performance characteristics with a much more efficient use of resources and time.\par
The developed methodology also benefited the optimization of realistic assignment and routing problems. The nonlinear range equation was used to estimate fuel efficiency and accurately estimate a maximum range and loiter time at a given altitude and specific performance characteristics. The use of this equation in optimization is a novel approach to involving performance characteristics within the assignment and routing formulation. Incorporating the nonlinear range equation in a simple priority assignment problem allowed for a more realistic view of how an aircraft might be assigned to interest locations for loiter depending on their performance characteristics and the priority assigned to each location. Additionally, the involvement of the nonlinear range equation in accounting for fuel efficiency in an a routing formulation for a lighter aircraft indicated that a realistic approach to aircraft performance may be better served with a dynamic constraint rather than using a constant for fuel burn in optimization formulations.
\section{Future Research}
This research is heavily constrained by the estimation of aircraft performance characteristics. Since velocity, lift over drag, and thrust-specific fuel consumption are dynamic as the aircraft burns fuels, accuracy could be increased with a table lookup for each major stage of the aircraft's weight change. The nonlinear range equation estimates the dynamic performance of the aircraft, but accuracy suffers as the weight changes drastically in the cases with a payload or external fuel tank drop. Another dynamic aspect of aircraft performance is the inclusion of flight conditions. While the NASIC model also does not include these factors, the inclusion of weather, maintenance, lethal envelopes, terrain constraints and/or uncertainty would elevate the real-world relevance of the developed methodology. \par
The research's optimization section focused on small example problems for validation of the formulations. Testing the formulated optimization models against relevant literature instances would increase the validity of the model and better examine the impact of adding the dynamic fuel constraint. Efficiency of the optimization model is decreased in its nonlinear form and would also benefit from a dynamic programming approach. \par
The inclusion of several different jet aircraft allowed for additional adjustments of the model. Extending the methodology to different types of aircraft to include propeller-driven aircraft and unmanned aerial systems would be a valid extension of this model. These are referenced by different range equations and parameters. The routing assignment problem might also better serve an unmanned aerial system versus a jet aircraft due to its mission capabilities and requirements such as range and loiter time over target areas.