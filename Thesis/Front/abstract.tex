\begin{abstract}
\hspace{0.5cm} This paper introduces a new method for determining tradeoffs between loiter time, range, and loiter altitude for jet aircraft utilizing specific aircraft performance characteristics and the Br\'eguet range and endurance equations. This method is applied to several routing assignment problem, taking advantage of the incorporation of performance characteristics in the formulation. Toy problems are generated using binary integer programming models and nonlinear mixed integer programming models. Results from this new methodology is compared to the existing model used by the National Air and Space Intelligence Center for flight planning. 

\end{abstract}
    

%     As an AFIT graduate student, you are about to write one of the
%     longest documents of your career.  Whether you use a ``what you see
%     is what you get'' (WYSIWYG) interface like Microsoft Word
%     \trademark or a typesetting system like \LaTeX\ldots when you
%     create a digital document, you are writing a program.  The larger
%     any program is, the more bugs it will contain and the more likely
%     the program will result in a catastrophic error.  Common bugs
%     result in improper formating, double words, lost sentences,
%     and---in the worst case---corrupted, unreadable files.
% 
%     Typesetting systems such as \LaTeX\ limit the new lines of code
%     generated with each new document.  As a result, they produce
%     cleaner, easier-to-debug products.  Additional benefits of \LaTeX\
%     are rapid reformatting and high quality typesetting of equations
%     and vector graphics.
