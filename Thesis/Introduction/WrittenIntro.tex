Probability of exceedance is a well-known field studied by many. The applicative aspects of knowing when a system is going to fail or what the probability is of exceeding a certain cost threshold is knowledge that most companies are willing to invest in. For example$\dots$\par
The growth of this field of study is testament to the sheer worth of theoretical exceedance probabilities for specific instances. Buffered probability of exceedance (bPOE) has even higher utility as it is a continuous probability that can interpolate threshold probability values without having to collect extra data. It is surprising that no studies have investigated the application of these buffered probability values on a network when so much information can be gained from calculating these values on an everyday optimization problem. Recently, Stan Uryasev has investigated the use of bPOE in studies of cost thresholds (see hurricane damage one) and developed starting work on the use of this tool in networks. However, further use can be developed on continuous networks, stochastic minimum cost network flow problems, and applications involving such.\par
The intent of this paper is to give an overview of the literature in the fields of network optimization, threshold networks, and buffered probability of exceedance. Most of the considerations will be directed toward a combination of these three fields in the latter parts of this paper and the literature review will focus on the wealth of literature which exists almost independently for all of these problem classes. In the following section, a brief introduction into the field is presented along with a general problem formulation, theoretical descriptions of each of the problem classes, and applications of these problems to set up the potential applications of this paper’s research. The paper will then delve into the methodology used and the heuristic developed to better solve for a networks bPOE. Additionally, there will be an application for the methodology, a practical example on real-world data, and the comparison of other techniques against the one developed in this paper. The paper will end with a discussion of areas for future research.\par
The attempt was made to include all relevant literature in the bibliography. Apologies?
