\section{Overview}
\hspace{.5cm} Range and endurance have been explored in the field of aeronautics by many individuals and companies, seeking to maximize both separately. However, since there is little utility for such use in a commercial environment, the trade-offs of range and endurance for an aircraft have not been widely explored. The National Air and Space Intelligence Center (NASIC) seeks an optimization tool that identifies the range and endurance trade-offs while minimizing error. Customers of NASIC are constructing aircraft mission plans and are looking for a spectrum of flight plans that meet their mission criteria. Analysts are currently using labor-intensive methods to provide range, radius, and time estimates for specific threat aircraft configurations. As a result, improved methods integrated into the process could result in significant time and cost savings. 

\section{Background}

\hspace{.5cm}The mission planning tool used by the National Air and Space Intelligence Center allows for a selection of aircraft, standardized load out, and flight profile. The outputs of the current tool are mission range or endurance depending on initial user inputs and, depending on the feasibility of those inputs, may output an error. The program mainly operates from a fuel constrained perspective and provides the outputs based on fuel reserve after user-defined parameters such as warm up, take-off, cruise altitude, length of combat, and fuel reserve. \par
The high fidelity of the mission planning tool allows for specific points to be found and then a trade-off of endurance and range is found by interpolating between these points which are at a close proximity. This aspect of finding the frontier of endurance and range is time consuming and must be done by hand by analysts familiar with the mission planning tool. The intent to find a trade-off is similar to Snyder \cite{Tradeoffs} finding the performance tradeoffs in various rotocraft. Some have used the simplification of equations to maximize range \cite{breguetRangeEqn, OptimizeBreguet}, while Raymer \cite{LoiterTimeFromRange} found methods of approximating endurance from range at specific points. Extensions of this problem in Alighanbari et al \cite{Alighanbari} formulated an optimization problems with endurance constructed as an objective and timing as the constraints of multiple UAV control. 

\section{Motivation}
\hspace{.5cm}The current method of deriving an optimal range and endurance for a customer with neither given as a parameter gives results that are accurate within five percent of actual, but are time consuming and inefficient. Additionally, the sensitivity of the current problem is unknown, given that the current method requires tabular calculations. The use of goal programming and optimization methods can be used to leverage the parameters in this dynamic problem and simplify the initial model. The geometric equivalence of the current solution method may be able to be used as a backbone for finding the vertical distance of two intersecting polynomials, representing aircraft's ingress to a range and egress to a base, using the result as an objective.\par
The primary objective of this paper is to describe an efficient tool to accurately find the frontier of optimal ranges and endurance for customers seeking options for their aircraft. A secondary objective is to find a full solution for all possible ranges and endurance with as few runs as possible, illustrating the trade-offs between a longer range or a longer loiter time. \par

